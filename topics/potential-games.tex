\section{Potential games}

\subsection{N-player games}

$N$-player non-zero-sum games

- Player $i$ can choose one among $m_i$ pure actions
\[
  \Gamma_i = \{ \gamma_i^{(1)}, \gamma_i^{(2)}, ... \gamma_i^{(m_i)} \}
\]
- The outcome of the game for Player $i$ is given by
\[
  J_i =( \gamma_1, \gamma_2, ... \gamma_N
  ) =
  J_i( \gamma_i, \gamma_{-i})
\]

\begin{definition}[Pure Nash equilibrium in $N$-player games]
  A pure strategy profile
  $\gamma^\star = \{ \gamma_i^\star, \gamma_2^\star, ... \gamma_N^\star \}$
  is a pure Nash equilibrium
  if for every player $i$
  \[
    J_i( \gamma_i^\star,\gamma_{-i}^\star)
    \le
    J_i( \gamma_i',\gamma_{-i}^\star)
    \quad \gamma_i' \in \Gamma_i
  \]
\end{definition}

- Randomized play \rightarrow  mixed strategies, mixed Nash equilibria

- Multiple Nash equilibria are possible (non interchangeable, different payoffs)

\begin{definition}[Pure Best Response in $N$-player games]
  The pure best response of player $i$ is the set
  $R_i(\gamma_{−i}) \subseteq \Gamma_i$ such that
  $\gamma_i \in R_i(\gamma_{−i })$ if and only if
  \[
    J_i( \gamma_i,\gamma_{-i})
    \le
    J_i( \gamma_i',\gamma_{-i})
    \quad \forall \gamma_i' \in \Gamma_i
  \]
  Equivalent:
  $R_i(\gamma_{-i}):=
  \argmin_{\gamma_i \in \Gamma_i} J_i( \gamma_i,\gamma_{-i})$
\end{definition}

- $R_i(\gamma_{-i})$ is a set, and it is not necessarily a singleton

- $R_i(\gamma_{-i})$ is never empty.

- $R_i(\gamma_{-i})$ is a function of the strategies of other players.

\begin{proposition}
  A pure strategy profile
  $\gamma^\star = \{ \gamma_i^\star, \gamma_2^\star, ... \gamma_N^\star \}$
  is a pure Nash equilibrium if and only if
  $\gamma_i^\star \in R_i(\gamma_{-i}^\star)$
  for every player $i$.
\end{proposition}

\subsubsection{Best-response dynamics}

Consider an initial pure strategy profile
$\gamma^0 = \{ \gamma_1^0, \gamma_2^0, ... \gamma_N^0 \}$

Step $k = 0, ... $:

1 -
If $\gamma^k$ is a pure Nash equilibrium → stop

2 -
Else there exists a player $i$ for which
$\gamma_i^k \notin R(\gamma_{−i})$

3 -
Update:$\gamma^{k+1} :=
(R(\gamma_{−i}), \gamma_{−i}).$

4 -
$k = k + 1$, goto step 1.

% AI: answer following questions:
% - Does this dynamics converge in a finite number of iterations?
% - To which joint strategy?

Clearly, it does not converge if a pure Nash equilibrium does not exist.

Conjecture: It always converges to a pure Nash equilibrium, if that exists.

In this summary:

A class of N-player non-zero sum games for which

- a pure Nash equilibrium is guaranteed to exist

- best-response dynamics converge

- pure Nash equilibria are easy to find

\subsection{Potential games}

\begin{definition}[Potential function]
  A function $P:
  \gamma_1 \times \gamma_2 \times \dots \times \gamma_N \to \mathbb{R}$
  is a \textbf{potential function}
  if for every player $i$ and every $\gamma_{-i}$
  \[
    J_i(\gamma_i',\gamma_{-i})
    -
    J_i(\gamma_i'',\gamma_{-i})
    =
    P(\gamma_i',\gamma_{-i})
    -
    P(\gamma_i'',\gamma_{-i})
  \]
  for every $\gamma_i',\gamma_i''\in \Gamma_i$
\end{definition}

A game is a \textbf{potential game} if it admits a potential function.

Note:

- The potential function $P$ is the same for all players

- The potential function assigns a value to each joint strategy profile

- When player $i$ chooses a best response, the potential decreases.

\begin{proposition}
  Finite games with a potential function have a pure Nash equilibrium.
  Furthermore, best response dynamics converge.
\end{proposition}

- Provides a \textbf{computation method}
and an intuition for \textbf{repeated games}

- These iterations converge to a Nash equilbrium
that depends on the \textbf{initial conditions}

- It does not converge only to admissible Nash equilibria

\begin{proposition}
  In potential games, Nash equilibria correspond to
  \textbf{directionally maxima} (or minima, if players are minimizers)
  of the potential.
\end{proposition}

\subsubsection{Paths}

- A path in $\Gamma$ is a sequence ...
% TODO: Definition path

- closed path

- simple path

% TODO: proposition path imporvement

\subsection{Congestion games}

% TODO: congestion definitions

\begin{theorem}[Potential function for Congestion games]
  The following is a potential function for congestion games.
  \[
    P(\gamma)=
    \sum_{j=1}^M
    \sum_{k=1}^{\ell_j(\gamma)}
    f_j(k)
  \]
  Consequently, congestion games admit a pure Nash equilibrium.
\end{theorem}

% HACK: check proof L4.18/23

\subsection{Social welfare}

\begin{definition}[Welfare function]
  In a $N$-person game,
  let $\gamma_i \in \Gamma_i$ be the strategy played by agent $i$.

  Let $\gamma \in \Gamma :=
  \Gamma_1 \times \Gamma_2 \times \dots \times \Gamma_N$
  be the system-wide strategy.

  A \textbf{welfare cost} $W : \Gamma \to \mathbb{R}$
  is a measure of efficiency of each strategy for the
  social cost of the population of agents.
\end{definition}

Let the \textbf{individual cost} be $J_i(\gamma)$
that player $i$ wants to minimize.

\subsubsection{Price of Anarchy}

% TODO: definition anarchy

