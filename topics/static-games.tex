\section{Static games}

\subsection{Basic Definitions}

Game theory studies mathematical models of conflict and cooperation
among rational decision-makers. A static game is characterized by:

\begin{itemize}
  \item \textbf{Players}: Rational agents making decisions.
  \item \textbf{Actions} (pure strategies): Finite sets of choices,
    e.g., $\Gamma = \{\gamma_1, \dots, \gamma_n\}$ for Player 1,
    $\Sigma = \{\sigma_1, \dots, \sigma_m\}$ for Player 2.
  \item \textbf{Information structure}: Simultaneous moves with no
    knowledge of opponents' choices.
  \item \textbf{Outcomes/Payoffs}: Cost functions $J_1(\gamma_i,
    \sigma_j) = a_{ij}$ (Player 1 minimizes) and $J_2(\gamma_i,
    \sigma_j) = b_{ij}$ (Player 2 minimizes), represented by matrices
    $A$ and $B$.
\end{itemize}

Examples include wireless power control (interference minimization),
Deer Hunt (coordination with multiple NE), self-driving cars at
intersections (Chicken game variants), and inspection games (no pure NE).

\subsubsection{Matrix Representation}

Games are often shown as bimatrix form:

\begin{center}
  \begin{tabular}{c|c|c}
    & $\sigma_1$ & $\sigma_2$ \\
    \hline
    $\gamma_1$ & $(a_{11}, b_{11})$ & $(a_{12}, b_{12})$ \\
    \hline
    $\gamma_2$ & $(a_{21}, b_{21})$ & $(a_{22}, b_{22})$ \\
  \end{tabular}
\end{center}

\subsubsection{Nash Equilibrium}

A pure Nash Equilibrium (NE) is a profile
$(\gamma^*, \sigma^*)$
where no player benefits from unilateral deviation:
\[
  J_1(\gamma^*, \sigma^*) \leq J_1(\gamma_i, \sigma^*) \forall i
\]
and
\[
  J_2(\gamma^*, \sigma^*) \leq J_2(\gamma^*, \sigma_j) \forall j
\]
Rational players converge to NE as stable outcomes.

\subsubsection{Dominated Actions}

Action $\gamma_k$ is strictly dominated by $\gamma_i$ if
\[J_1(\gamma_i, \sigma_j) < J_1(\gamma_k, \sigma_j) \forall j\]
Rational players avoid dominated actions.

\subsubsection{Reduced Game}

Iteratively remove dominated actions to simplify the game, revealing NE.

\subsubsection{Security Levels and Policies}

Pure \textbf{security level} for Player 1:
\[
  \underline{J_1} = \min_i \max_j a_{ij}
\]
(worst-case guaranteed cost assuming adversarial opponent)

\textbf{Security policy}
\[
  \arg\min_i \max_j a_{ij}
\]

\subsection{Multiple Nash Equilibria}

Games may have multiple pure NE, not interchangeable

\textbf{Partial order:}

$(J_1', J_2') \prec (J_1, J_2)$
if $J_1' \leq J_1$, $J_2' \leq J_2$
(at least one strict)

\subsubsection{Admissible Nash Equilibria}

A NE is admissible if not Pareto-dominated by another NE
(better for one, not worse for other).
Represent via poset: outcomes ordered by Pareto dominance.

\textbf{Hasse diagram}
visualizes this poset as a graph where nodes are NE outcomes,
edges indicate dominance (from worse to better),
and minimal elements (with no incoming edges from below) are admissible,
representing undominated equilibria that rational players might prefer.

\subsection{Mixed Strategies}

\textbf{Extend to probabilities}
$y \in \Delta^n$ (simplex), $z \in \Delta^m$

\textbf{Expected costs}
$J_1(y,z) = y^\top A z$, $J_2(y,z) = y^\top B z$.
Distinguish from
pure strategies (deterministic).

\subsubsection{Security Levels}

\textbf{Mixed security level}
$\min_y \max_z y^\top A z = \min_y \max_j (A^\top y)_j$

(linear program, computable efficiently).

\textbf{Mixed security strategy:}
$\arg\min_y \max_z y^\top A z$

(robust against worst-case mixed opponent, often lower than pure level)

\textbf{Computational complexity}
Polynomial-time solvable via LP, unlike some NE computations.

\subsubsection{Mixed Nash Equilibrium}

Best responses $(y^*, z^*)$ where:
\[
  y^\top A z^* \geq (y^*)^\top A z^* \forall y
\]
\[
  (y^*)^\top B z \geq (y^*)^\top B z^* \forall z
\]
Certify by checking pure deviations.

\textbf{Completely mixed}
all probabilities positive, satisfy indifference:
\[
  A z^* = p^* \mathbf{1} \quad (y^*)^\top B = q^* \mathbf{1}^\top
\]
\textbf{Non-completely mixed}
exist without indifference

e.g., 3x3 game with $y^* = z^* = (1/3,2/3,0)$

\subsection{Nash Theorem}

\begin{sstFullFrame}
  \center\textcolor{white}{\textbf{
      Every finite game has at least one mixed NE
  }}
\end{sstFullFrame}

\textbf{Proof} via Kakutani's fixed-point theorem:
The best-response correspondence
$\beta: \Delta^n \times \Delta^m \to 2^{\Delta^n \times \Delta^m}$
maps each strategy profile to the set of best responses
(non-empty, convex, upper hemicontinuous due to continuity of
expected payoffs and compactness of simplices)
Kakutani guarantees a fixed point, which is a NE where each player's
strategy is a best response to the other's.

