\section{Convex games}

\subsection{Games with infinite actions}

Examples:
\begin{itemize}[-]
  \item
    money (e.g., auctions)
  \item
    physical control inputs (force, velocity, ...)
  \item
    waiting time (e.g., Start-Stop)
  \item
    coverage path of a surveillance camera
\end{itemize}

\subsubsection{Problems with infinite actions}

\textbf{Theory}
Most of the properties and results that we saw don’t hold anymore!

\begin{itemize}[-]
  \item
    every time there is an argmin or argmax
  \item
    Nash Theorem
\end{itemize}

\textbf{Algorithms}
The few algorithms that we have seen
are also not suited for infinite actions

\begin{itemize}[-]
  \item
    algorithms \textit{by inspection}
  \item
    linear programming
\end{itemize}

\subsubsection{Example: Cournot competition}

\begin{sstTitleBox}{Problem Setup - Cournot competition}
  \begin{sstOnlyFrame}
    Consider two producers competing in a market.

    Each player (producer) $i = 1, 2$ decides on the
    \textbf{quantity} to produce denoted by $x_i \ge 0$,
    and has a production marginal cost of $c > 0$.

    The market price $p :  \mathbb{R}\times\mathbb{R}\rightarrow\mathbb{R}$
    is a linearly decreasing function of the
    total production $x_1 + x_2$
  \end{sstOnlyFrame}
  \begin{sstOnlyFrame}
    \[p(x_1, x_2) = a − b(x_1 + x_2)\]
  \end{sstOnlyFrame}
\end{sstTitleBox}

% TODO: Example: Cournot competition
\begin{enumerate}
  \item
    Write each producer’s losses as a function of the quantities produced.
  \item
    Derive the Nash equilibrium.
  \item
    What is the Nash equilibrium if producers’ decisions are
    restricted to $[0, k]$?
\end{enumerate}

\subsubsection{Example: Betrand competition}

\begin{sstTitleBox}{Problem Setup - Betrand competition}
  \begin{sstOnlyFrame}
    Consider two producers competing in a market.

    Each producer decides on the price $x_i \ge 0$ of their product,
    and has a production marginal cost of $c > 0$.
  \end{sstOnlyFrame}
  \begin{sstOnlyFrame}
    The total demand is 1 unit and the consumers choose to buy
    from the producer with the lowest price
    (if both firms declare the same price, then half of the demand
    chooses firm 1 and the other half chooses firm 2).
  \end{sstOnlyFrame}
\end{sstTitleBox}

% TODO: Example: Betrand competition
\begin{enumerate}
  \item
    Write each producer’s losses as a function of the price they charge.
  \item
    Derive the Nash equilibrium.
  \item
    What is the Nash equilibrium if each producer has the capacity
    to serve maximum 2/3 of the unit demand?
\end{enumerate}

\subsubsection{Definitions and background}

% TODO: definitions convexity
Convexity

- Convex set
- Convex function

Differentiable functions

- Convex function
- Characterization: Hessian

\subsection{Convex games}

N-player game with continuous action spaces
\begin{itemize}[-]
  \item
    $N$ player game
  \item
    Player $i$’s action $x_i \in K_i \subset \mathbb{R}^{n}$
  \item
    $K_i$ is non-empty, closed and convex
  \item
    def: $K = K_1 \times K_2 \times \dots \times K_N ⊂ \mathbb{R}^{nN}$
  \item
    Player $i$’s outcome $J_i: K \to \mathbb{R}$
  \item
    Compact notation: $J_i (x_i, x_{−i})$
\end{itemize}

\begin{definition}[Nash Equilibrium - Convex Games]
  mostly the same as usual
  % TODO: NE convex games
\end{definition}

\begin{theorem}[Existance of Pure NE - Convex Games]
  Consider an $N$-player game with continuous action spaces $K_i$.
  Suppose
  \begin{itemize}[-]
    \item
      action spaces $K_i \subset \mathbb{R}^{n}$
      are \textbf{compact and convex}
    \item
      $J_i$ are \textbf{continuous} in $x \in K$
    \item
      $J_i$ are \textbf{convex} in $x_i$ for fixed $x_{−i}$ .
  \end{itemize}
  Then a pure Nash equilibrium $x^\star$ exists.
\end{theorem}

Games that satisfy those conditions are called \textbf{convex games}.

\begin{theorem}[Maximum theorem]
  % NOTE: maximum theorem, upper hemi-continuity
\end{theorem}

Upper hemi-continuity

% TODO: apply convex game theorem to cournot, bertrand
Cournot and Bertrand models

\subsection{Variational inequalities}

\begin{definition}[Variational inequalitiy]
  % TODO: definition Variational inequality
\end{definition}

VI and convex optimization

\begin{definition}[First-order optimaliy conditions]
  % TODO: VI, SOL(KF) to optimization
\end{definition}

\subsubsection{VI and Nash Equilibria}

Characterization of Nash Equilibria of convex games

\begin{definition}[Nash Equilibria and Variational Inequality]
  % TODO: VI, SOL(KF) NE
\end{definition}

Two important advantages coming from the connection between
\textbf{variational inequalities} and \textbf{Nash equilibria}.
\begin{itemize}[-]
  \item
    Borrow \textbf{theoretical results} from VI (e.g. uniqueness)
  \item
    Borrow \textbf{numerical methods} to solve VI / find NE.
\end{itemize}

Monotone maps

\begin{definition}[Monotonicity]
  % TODO: monotonicity
\end{definition}

How to check monotonicity of a map $F$?
% TODO: How to check monotonicity of a map $F$?

\subsubsection{Uniqueness of Nash Equilibria}

\begin{definition}[SOL$(K,F)$ is a singleton]
  % TODO: SOL(K,F) singleton
\end{definition}

\begin{corollary}

\end{corollary}

% TODO: example L5.21/26

\subsection{Computing the Nash Equilibrium of a convex game}

% TODO: example L5.22/28

Learning/computing the Nash Equilibrium

% TODO: learning NE
%
\begin{definition}[Best-response iteration]
\end{definition}

Iterative NE-seeking algorithm

\begin{definition}[Iterative update]
\end{definition}

\subsubsection{Projected game map}

\begin{definition}[Conjecture]
\end{definition}

Interpretation

Analysis

\begin{definition}[Assumptions]
\end{definition}

\subsubsection{Analysis of convergence}

Equilibrium

\begin{proposition}
\end{proposition}

Contractive maps

\begin{theorem}[Banach Fixed Point]
\end{theorem}

Contractiveness of the projected-game-map iteration

Projection is non-expansive

\begin{theorem}[Projection non-expansive]
\end{theorem}

Convergence result

\begin{theorem}[Convergence of projection]
\end{theorem}

Example TCP congestion

