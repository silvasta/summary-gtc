\section{Convex games}

\subsection{Games with Infinite (Continuous) Action Spaces}
Many real-world games involve continuous actions
(e.g., quantities in auctions, prices, control inputs like
force/velocity, waiting times, coverage paths).

\textbf{Challenges}
Finite-game results
(e.g., Nash's theorem, argmin/argmax existence)
fail;
algorithms like inspection or linear programming are unsuitable.

\subsubsection{Motivating Examples}
\begin{sstTitleBox}{Cournot Competition}
  \begin{sstOnlyFrame}
    $N$ producers: Choose quantities $x_i \in [0,k]$.\\
    Marginal cost: $c > 0$.\\
    Market price: $p(x) = a - b \sum x_j$ ($a,b > 0$).\\
    Losses: $J_i = c x_i - p(x) x_i$.\\
    Properties: Convex game; unique NE.\\
    Best response:
    \[
      x_i^* = \max\{0, \min\{k, (a-c - b \sum_{j\neq i}
      x_j^*)/(2b)\}\}
    \]
    Potential:
    \[
      P(x) = (c-a)^\top x + \frac{1}{2} x^\top (b I + b
      \mathbf{1}\mathbf{1}^\top) x
    \]
    Minimizing $P$: Yields the NE.
  \end{sstOnlyFrame}
\end{sstTitleBox}

\begin{sstTitleBox}{Bertrand Competition}
  \begin{sstOnlyFrame}
    Producers: Choose prices $x_i \geq 0$.\\
    Marginal cost: $c > 0$.\\
    Demand: Splits to lowest price (tie: half each).\\
    Losses: Discontinuous and non-convex.\\
    Equilibrium: No pure NE without capacities.\\
    With capacity $2/3$: Still no pure NE (mixed NE exists).\\
    Properties: Not convex if action sets unbounded.
  \end{sstOnlyFrame}
\end{sstTitleBox}

\subsection{Definitions and Background}

\textbf{Convex Set}: Closed under convex combinations.

\textbf{Convex Function}: $f(\lambda x + (1-\lambda)y) \leq \lambda
f(x) + (1-\lambda) f(y)$, $\lambda \in [0,1]$.
Strictly convex if $<$.

For differentiable $f$:
Convex iff $\nabla^2 f \succeq 0$ (Hessian PSD)

First-order: $f(y) \geq f(x) + \nabla f(x)^\top (y-x)$.

\textbf{Convex Game}:
$N$ players

Actions $x_i \in K_i \subseteq \mathbb{R}^n$ (compact, convex, nonempty)

Losses $J_i(x_i, x_{-i}): K \to \mathbb{R}$ \\
continuous in $x$ convex in $x_i$ (fixed $x_{-i}$)

\textbf{Pure Nash Equilibrium (NE)}:
$x^* \in K$ s.t. $\forall x_i \in K_i$, all $i$.
\[
  J_i(x_i^*, x_{-i}^*) \leq J_i(x_i, x_{-i}^*)
\]

Assumptions sufficient but not necessary for existence; relaxing
(e.g., open/unbounded/non-convex $K_i$) may destroy NE
(counterexamples: open intervals lead to boundary deviations;
unbounded allows infinite descent).

\subsection{Existence of Pure NE}

\begin{theorem}[Existence of Pure NE - Convex Games]
  If $K_i$ compact convex, $J_i$ continuous, convex in $x_i$, then
  $\exists$ pure NE.

  \textbf{Proof}: Best-response map
  $\Omega(x) = (\arg\min_{x_1} J_1(x_1, x_{-1}), \dots)$
  is upper hemicontinuous, convex-valued. Kakutani fixed-point applies.
  (Compactness crucial; non-convex $K_i$ may still have NE, e.g.,
  Prisoner's Dilemma.)
\end{theorem}

\subsection{Variational Inequalities (VI)}

\textbf{Game Map}
$F(x) = \nabla_{x_1} J_1(x), \dots, \nabla_{x_N} J_N(x)$

\textbf{VI$(K,F)$}
Find $x^* \in K$ s.t.
$F(x^*)^\top (y - x^*) \geq 0$ $\forall y \in K$.
Equivalent to first-order optimality in convex optimization.

\textbf{Theorem (NE $\iff$ VI)}: In continuously differentiable
convex games, $x^*$ is NE iff solves VI$(K,F)$.
Advantages: Borrow VI theory (uniqueness) and numerics (solvers) for NE.
Applicability requires $J_i$ continuously differentiable (e.g., fails
for non-smooth like max\{0, $x_i$\}).

\subsection{Monotonicity and Uniqueness of NE}

\textbf{Monotonicity of $F$}: Monotone if $(F(x)-F(y))^\top (x-y)
\geq 0$; strictly if $>0$ ($x \neq y$); strongly if $\geq \mu
\|x-y\|^2$, $\mu>0$.
Check via Jacobian: $F$ (strictly/strongly) monotone if $\nabla F +
(\nabla F)^\top \succ 0$ (or $\succeq 0$ for monotone).

\textbf{Theorem (Uniqueness)}: If $F$ strictly monotone on $K$, then
unique NE (SOL(VI$(K,F)$) singleton). Strong monotonicity implies
uniqueness globally.

\textbf{Warning}: Individual $J_i$ strictly convex $\not\Rightarrow$ unique NE
(counterexample: $J_i = (x_1 - x_2)^2$ both; $F$ not monotone).
Exam-relevant: Cournot Jacobian positive definite $\implies$ strong
monotonicity $\implies$ unique NE.

\subsection{Projected Game Map Dynamics}

Computing/Learning NE

\textbf{Discrete-Time Iteration}: $x(t+1) = \Pi_K [x(t) - \gamma
F(x(t))]$, or per-player: $x_i(t+1) = \Pi_{K_i} [x_i(t) - \gamma
\nabla_{x_i} J_i(x(t))]$.
Interpretation: Projected gradient step assuming others fixed
(best-response-like with conjecture).

\textbf{Assumptions for Convergence}: $F$ $\mu$-strongly monotone,
$L$-Lipschitz ($\gamma \in (0, 2\mu / L^2)$).

\textbf{Theorem (Convergence)}: Under assumptions, iteration
converges to unique NE (Banach fixed-point: map contractive).
Proof: Projection non-expansive ($\Pi_K$ 1-Lipschitz); strong
monotonicity + step-size ensure contractivity.

In potential convex games (e.g., Cournot), best-response dynamics
converge to NE (global min of potential).
Applications: TCP congestion control, power control (SINR games),
ride-hailing pricing

\examBox{exam-like}{oligopoly with substitutability $\theta$)}

