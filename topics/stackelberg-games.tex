\section{Stackelberg games}

\subsection{Definition of Stackelberg games}

% NOTE: example page 1

\begin{itemize}
  \item
    $\Gamma$: pure-strategy space of the \textbf{leader} (Player 1)
  \item
    $\Sigma$: pure-strategy space of the \textbf{follower} (Player 2)
\end{itemize}

\begin{sstTitleBox}
  {Rational reaction set}
  % TODO: Rational reaction set
\end{sstTitleBox}

\begin{sstTitleBox}
  {Stackelberg equilibrium}
  \begin{sstOnlyFrame}
    A pair of strategies $\tilde{y} \in \mathcal{Y}$
    and $\tilde{\sigma}(y) : \mathcal{Y} \to \Sigma$
    is a Stackelberg Equilibrium if
    \begin{itemize}
      \item Player 1 plays the best response to $\tilde{\sigma}(y)$:
        % TODO: finish with formulas
      \item Player 2 plays the best response to $y$:
    \end{itemize}
  \end{sstOnlyFrame}
\end{sstTitleBox}

Applications

\subsection{Stackelberg zero-sum games}

Stackelberg Equilibrium in zero-sum games

% TODO: Stackelberg EQ zero-sum

\subsubsection{Security strategies}

\begin{sstTitleBox}
  {Mixed security strategy}
\end{sstTitleBox}
% TODO: security strategies Stackelberg
\begin{sstTitleBox}
  {Pure security strategy}
\end{sstTitleBox}

\subsubsection{Stackelberg vs Nash}

In zero-sum games,
mixed Stackelberg equilibria and Nash equilibria coincide.
% NOTE: extend maybe?

\subsection{Stackelberg non-zero-sum games}

% TODO: Stackelberg, Highest, Lowest

\begin{sstTitleBox}
  {Highest leader cost}
\end{sstTitleBox}

\begin{theorem}[Upper bound on Stackelberg cost]

\end{theorem}

\begin{theorem}[Nash vs Highest Leader Cost]

\end{theorem}

\begin{sstTitleBox}
  {Lowest leader cost}
\end{sstTitleBox}

\begin{theorem}[Nash vs Lowest Leader Cost]

\end{theorem}

Generic Stackelberg games

\subsubsection{Computation of the Stackelberg equilibrium}

\begin{sstTitleBox}
  {Divide-and-conquer algorithm}
\end{sstTitleBox}

\begin{sstTitleBox}
  {Linear programming}
\end{sstTitleBox}

In non-zero-sum games with mixed strategies,
computing Stackelberg Equilibria is
much easier than computing Nash Equilibria!

\subsection{Security games}

% TODO: security games
Randomized defender stategy in security games

\begin{definition}[Coverage vector]
\end{definition}

Randomized attacker stategy in security games

\subsubsection{Stackelberg solution of a security game}

\subsubsection{Nash equilibria of security games}

Auxiliary zero-sum security game

Interpretation

Stackelberg equilibria are Nash equilibria

