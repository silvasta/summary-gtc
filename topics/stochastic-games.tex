\section{Stochastic games}

- Understand shortcoming of randomized feedback games for modelling
large or infinite number of stages.

- Derive stochastic games as a tractable class of randomized feedback games.

- Construct state transition models for stochastic games.

- Contrast solution methods in finite stage vs. infinite stage stochastic games.

- Bonus: Recognize curse of dimensionality in stochastic games with
large number of players and introduce stochastic population games.

\subsection{Epidemic game}

% \begin{center}
%   \begin{tikzpicture}[
%       % Global styling for the tree
%       tree/.style={ % Style for edges
%         -Latex, % Arrow style (from arrows.meta)
%         thick,
%         draw=gray,
%       },
%       level/.style={ % Spacing between levels
%         level distance=1.5cm, % Vertical distance between levels
%         % (adjust as needed)
%         % sibling distance=3cm, % Horizontal distance between siblings
%         % (adjust for wider/narrower tree)
%       },
%       root node/.style={ % Style for root node
%         draw=blue, % Border color
%         fill=blue!20, % Fill color
%         circle, % Shape (circle; change to rectangle, etc.)
%         minimum size=1cm, % Size
%         font=\bfseries, % Font
%       },
%       level-1/.style={ % Style for first-level nodes
%         draw=green,
%         fill=green!20,
%         rectangle,
%         minimum width=2cm,
%         minimum height=0.8cm,
%         font=\small\bfseries,
%       },
%       level-2/.style={ % Style for leaf/second-level nodes
%         draw=red,
%         fill=red!20,
%         ellipse, % Shape for leaves
%         minimum width=1cm,
%         sibling distance=1cm, % Horizontal distance between siblings
%         minimum height=0.6cm,
%         % sibling distance=1cm,
%         font=\small,
%       },
%       every node/.style={ % Default for all nodes
%         align=center, % Center text
%       },
%     ]
%
%     % The tree structure starts here
%     \node[root node] (root) {Alice}
%     child { node[level-1] (n1) {H}
%       child { node[level-2] {H} }
%       child { node[level-2] {O} }
%     }
%     child { node[level-1] (n2) {O}
%       child { node[level-2] {H} }
%       child { node[level-2] {O} }
%     };
%
%   \end{tikzpicture}
% \end{center}

\subsubsection{1 stage}

Consider the case with only 1 stage or day.

- What is the corresponding tree?

- Is it a feedback game?

- What is the Nash equilibrium strategy for Alice and for Eve?

- What is the value of the game for Alice and for Eve?

\subsubsection{2 stages}

Consider the case where there are 2 stages or days.

- Draw the corresponding game tree

- Is it a feedback game?

- Solve the game. How many LP did you have to solve?

\subsubsection{Taming the complexity}

- subgames dependent of states \{(NI,I),(I,I)\}

- from $4^{k-i}$ to $2K$

\subsection{State transition model}

- \textbf{Stages} become \textbf{discrete time}

- \textbf{Behavioral strategies}
(maps from information sets to actions)
become \textbf{feedback policies}
(maps from states to inputs)

- \textbf{Kuhn’s theorem holds}:
NE in behavioral strategies guaranteed to exist

- Can compute (subgame-perfect) behavioral NE
using \textbf{backward induction}

\subsection{Probabilistic state transitions}

- from certainty $1$ to probability $\alpha$

\subsection{Stochastic game}

\begin{sstTitleBox}
  {Basic elements}
  \begin{sstOnlyFrame}
    \begin{itemize}
      \item Stages $k = 0,\dots, K−1$
      \item State space
        $x_k \in \mathcal{X} = \{x^1,\dots, x^\ell\}$,
        initial state $x_0$
      \item Action space
        $u_k \in \mathcal{U} = \{u^1,\dots,u^n\}$,
        $v_k \in \mathcal{V} = \{v^1,\dots,v^m\}$
      \item Stage outcome functions
        $g^{(1)} (x_k , u_k , v_k )$,
        $g^{(2)} (x_k , u_k , v_k )$
      \item State transition probabilities
        $\mathbb{P}(x_{k+1} | x_k , u_k , v_k )$
    \end{itemize}
  \end{sstOnlyFrame}
\end{sstTitleBox}

\subsubsection{Finite stage stochastic game}

- Behavioral strategies

- \textbf{Outcome}
$\mathbb{E}[\sum g | x_0]$

\subsubsection{Infinite stage stochastic game}

- \textbf{Stationary} Behavioral strategies

- \textbf{Outcome}
$\mathbb{E}[\sum \delta^k g | x_0]$

\subsubsection{Value function}

- seperable to stage cost + next V

\subsubsection{Solving stochastic game}

- Finite stage stochastic game: Backward induction

- Infinite stage stochastic game: Bellman equation

\subsubsection{Example}

- 2 car at crossing, go wait, Exam Question
\[
  \begin{matrix}
    (100,100)&(0,1)\\
    (1,0)&(v,v)
  \end{matrix}
\]
- solve by hand with p12gw $y^T G z$

\subsection{Summary: Stochastic games}

\section{Bonus Material}

Unfortunate news: Even ($\ell$) single-stage games might not be enough!

Epidemic game revisited:

With Alice and Eve, we had 4 states,
adding 1 player results in 8 states.

$2^N$ with $N$ Number of players

\textbf{Stochastic population games}

Assumption 1

There is a large number of players $N\to\infty$

Assumption 2

The state is separable in the players

Assumption 3

Players are anonymous. Their stage outcomes and state transition
probabilities depend on own state and action, and distribution of
others’ states and actions

Assumption 4

Players are symmetric, i.e., they have identical stage outcome
functions and state transition probabilities.

Assumption 5

The stage outcome function and state transition probabilities
are continuous in the state and action distributions

Theorem

Under Assumptions 1–5, there exists a stationary state and action
distribution pair $(d^\star,\pi^\star)$ that constitutes a Nash equilibrium
for the $\delta$-discounted infinite stage stochastic population game.

We have not covered

- Games with partial information / Bayesian games

- Differential games

- ...

