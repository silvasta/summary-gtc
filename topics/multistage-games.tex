\section{Multistage games}

Multistage games extend repeated games by allowing variable stages,
sequential or simultaneous moves, imperfect information, and memory
constraints.

The extensive form (game tree) is preferred over the
normal (matrix) form due to efficiency, especially for large games
like tic-tac-toe (9! pure strategies for Player 1 in simplified form).

\subsection{Extensive Form Representation}

\textbf{Game tree}
Root to leaves,
nodes are decision points labeled by players,
edges are actions,
leaves have payoffs
(focus on zero-sum for saddle-points)

\textbf{Information sets}
Group indistinguishable nodes (dashed lines)
each set must have identical action sets.
Imperfect information arises when sets span multiple nodes.

\textbf{Examples}
Tic-tac-toe (symmetries create large info sets)
surprise quiz paradox (teacher-student game with surprise payoffs).

\textbf{Conversion to normal form}
Enumerate pure strategies to form payoff matrix,
but size explodes exponentially.

\subsection{Strategies in Extensive Form}

\textbf{Pure strategy}
Maps each info set to one action.

\textbf{Mixed strategy}
Probability distribution over pure strategies
(dim: $\prod |U_i| - 1$, exponential).
Randomizes ex-ante over full plans.

\textbf{Behavioral strategy}
Independent distribution over actions per info set
(dim: $\sum (|U_i| - 1)$, linear).
Local randomization during play.

\textbf{Kuhn’s theorem} (feedback games)
Every mixed strategy has an equivalent behavioral strategy
inducing the same outcome distribution and payoffs.

\subsection{Equilibria and Credibility}

Many Nash equilibria (NE) exist
but may rely on non-credible threats
(e.g., nuclear deterrence: "launch if provoked" only credible if automated).

\textbf{Subgame Perfect Equilibrium (SPE)}
Strategy profile that is NE in every subgame
(refines NE by ensuring credibility).

Zermelo’s theorem (perfect info, finite zero-sum, no chance): One
player has winning strategy or draw.

\subsection{Backward Induction}

For perfect info games:
Start at leaves,
propagate optimal choices upward,
yields pure SPE efficiently (vs. matrix enumeration).

Applications:
Chess (theoretical draw/win, unsolved),
tic-tac-toe (draw with optimal play).

\textbf{Feedback games}
Each Player 1 info set roots a separate subgame,
enables recursive decomposition.

In feedback games with behavioral strategies:

- At each info set, solve local matrix game for minimax value and
behavioral NE (via LP if needed).

- Propagate values backward to root; yields SPE in behavioral strategies.
No convergence guarantee for pure NE, but Nash’s theorem ensures
mixed/behavioral existence in finite games.

\subsection{Relevant Examples and Exercises}

\textbf{Surprise test paradox}
Modeled as non-feedback zero-sum game (K=2 days)
teacher maximizes surprise, student minimizes.

NE: Teacher tests day 1 w.p. 1/3, student studies day 1 w.p. 2/3,
average study days 4/3 >1 (incentivizes more study).
Backward induction fails due to imperfect info,
but NE shows surprise possible.

- \textbf{Tic-tac-toe}: Extensive form has $9 \times 7^8 \times 5^6
\times 3^4 \times 1^2$ pure strategies for P1; backward induction
proves optimal draw.

\examBox{Exam focus}
{Distinguish mixed vs. behavioral (equivalence in feedback games);
  apply backward induction for SPE in zero-sum extensive/feedback
  games; analyze credibility in non-zero-sum settings; model
  applications like epidemics or power control as stochastic multistage
games with state-dependent equilibria. }

