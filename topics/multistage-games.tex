\section{Multistage games}

- Example: Tic-tac-toe

“Advanced” games

Features:
\begin{itemize}
  \item multiple stages
  \item different order of choice
  \item variable number of stages
  \item partial information (dependent on actions)
  \item memory-constrained players
\end{itemize}

The (repeated) \textbf{matrix form} is not the most effective representation.

\subsection{Tree representation
}
\begin{itemize}
  \item The game evolves from the root to the leaves
    \begin{itemize}
      \item Let us consider zero-sum games for now
    \end{itemize}
  \item Each level of the tree corresponds to a player’s turn
  \item Links correspond to actions
  \item Each leaf is associated to a final outcome
  \item Nodes of each player are divided into information sets
    \begin{itemize}
      \item each node in the same information set has the same branches
    \end{itemize}
\end{itemize}

Actions and strategies

% HACK: information  set
% TODO: action and strategies multistage tree

- Simultaneous play

- Sequential play

From extensive form to matrix form

% TODO:

Games in extensive form can be reformulated in matrix form!

Credible threats

% TODO:

\subsubsection{Subgame perfect equilibria}

A strategy is a subgame perfect equilibrium
if it represents a NE of every subgame of the original game.

- The notion of subgame is not always well defined.

- One special case: games with perfect information.

Example: Chess

\subsubsection{Backward induction}

Features

- More efficient than exploring the matrix form

- Returns a strategy that is a subgame perfect equilibrium

Can we always apply backward induction?

We can for games with perfect information.

Feedback games

A multi-stage game in extensive form is a feedback game
if each “Player 1” node is the root of a separated sub-game.

Backward induction in feedback games

Starting from the leaves, identify subgames for which we can determine the pure
NE strategy...

... either because players have full information...

... or because they play a simultaneous game with pure NE.

\begin{itemize}
  \item Solve the game from the leaves towards the root
  \item Move up stage-by-stage (not level-by-level)
  \item Record the pure equilibrium strategy for each information set
  \item If the algorithm converges to the root, then we have a pure NE.
  \item No guarantees of convergence, even when a pure NE exists.
\end{itemize}

Towards randomized strategies

% TODO: image, aussagen, L8.33/59

\subsection{Mixed strategies}

% TODO: mixed

Nash’s Existence theorem
% HACK: Nash’s Existence theorem

\subsection{Behavioral strategies}

% TODO: pure, mixed Behavioral

Mixed vs. behavioral

Kuhn’s theorem

\subsection{Single stage game}

How to search for a Nash equilibrium behavioral strategy?

Step 1-3

Backward induction

