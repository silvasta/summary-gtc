\section{Auctions}

Auctions model strategic interactions where $N$ bidders (players)
submit bids $x_i \geq 0$, with private true values $t_i$ for the
item(s). The outcome is determined by winner selection $w(x)$ and
payment $p(x)$, yielding utility for bidder $i$:
\[
  J_i(x) =
  \begin{cases}
    t_i - p(x) & \text{if } i = w(x) \\
    0 & \text{otherwise}.
  \end{cases}
\]
(Note: Utilities are typically defined as gains, so we maximize $U_i
= -J_i$; lectures use cost minimization.)

\subsection{First-Price Sealed-Bid Auction}

Winner: $w(x) = \arg\max_i x_i$; pays $p(x) = x_{w(x)}$.

- Overbidding ($x_i > t_i$) is strictly dominated by truthful bidding
($x_i = t_i$): winning with overbid risks loss if $p > t_i$.

- Truthful bidding is not dominant: best response depends on others'
bids; rational players underbid ($x_i < t_i$) to balance win
probability and surplus.

- No dominant strategy equilibrium; Bayes-Nash equilibria require
valuation distributions (e.g., symmetric independent private values
yield shading: $x_i = \frac{N-1}{N} t_i$ for uniform [0,1]).

Private information ($t_i$ unknown) leads to inefficiency; equilibria
are not socially optimal.

\subsection{Second-Price Sealed-Bid Auction (Vickrey)}

Winner: $w(x) = \arg\max_i x_i$; pays $p(x) = \max_{j \neq w(x)} x_j$.

- \textbf{Dominant Strategy}: Truthful bidding ($x_i = t_i$) is weakly dominant.

\textbf{Proof}:
Bid $x_i$ affects only win/loss, not payment (fixed at second-highest).
Overbidding risks winning unprofitably,
underbidding risks losing profitably.
Thus, $U_i(t_i, x_{-i}) \geq U_i(x_i, x_{-i})$ for all $x_i, x_{-i}$

- \textbf{Incentive Compatibility (IC)}
Truth-telling optimal regardless of others.

- \textbf{Social Efficiency}: Allocates to highest $t_i$ if truthful.

- Revenue: Equals second-highest $t_i$; not maximal but elicits truthfulness.

- IC and efficiency often align: revealing $t_i$ enables optimal allocation.

\subsection{Generalized auctions}

\subsubsection{Bids}
Each \textbf{bid} is represented by a pair
$x_j = (b_j, m_j)$

- $b_j$ is the bidded amount

- $m_j$ describes the object of the bid

(Can be extended to allow multiple bids)

\textbf{Fungible goods}
$m_j\in\mathbb{R}_{>0}$
parts of a total quantity $M$

\textbf{Non-fungible goods}
$m_j\in 2^\mathcal{M}$
with finite set of items $\mathcal{M}$

\subsubsection{Choice function}

Choice function $w$ maps bids $x$ into
$N$-dimensional binary vector
\[w_j(x) =
  \begin{cases}
    1 & \text{if bid }j\text{ is accepted}\\
    0& \text{otherwise}
  \end{cases}
\]

- Choice constraints

\subsubsection{Payment function}

Payment function $p$ maps bids $x$ into
$N$-dimensional vector where $p_j(x)$
is the payment requested from the player
that placed the bid $j$

\subsection{Vickrey-Clarke-Groves (VCG) Mechanism}

Generalizes Vickrey for complex allocations.

\subsubsection{VCG Choice}

$w^*(x) = \arg\max_{w: (w,m) \in C} \sum_j b_j
w_j$ (maximizes reported social welfare).

\subsubsection{VCG Payment (Clarke pivot)}
\[
  p_j(x) =
  \max_{w: w_j=0} \sum_{k \neq j} b_k w_k
  - \sum_k b_k w_k^*(x) - b_j w_j^*(x)
\]
\textbf{Interpretation}
$p_j$ is the welfare loss imposed on others by
including bid $j$ (externality).

\textbf{Social Utility}: $U(t,w) = \sum_i t_i w_i$

(aggregate true value; payments cancel out).

If truthful ($b_j = t_j$), $w^*$ maximizes $U^*$.

\subsubsection{Properties}

\textbf{Truthful Bidding Dominant}
Weakly dominant strategy

Proof: payment independent of own bid; misreporting can only worsen
allocation relative to truthful externality.

\textbf{Non-Negative Utility (Individual Rationality)}

Truthful bidding yields $U_i \geq 0$

Proof: $U_i = U^*(t) - U^*_{-i}(t_{-i}) \geq 0$,
as full optimum $\geq$ optimum without $i$

\textbf{Social Efficiency}: Achieves max $U$ if truthful.

\textbf{DSIC}: Dominant-strategy incentive compatible.

\examBox{Exam-Relevant Insights from Exercises}{
  \textbf{Collusion Vulnerability}:
  Shill bidding (e.g., splitting bids)
  can manipulate outcomes, increasing payments
  (e.g., single player posing as multiple to extract higher revenue in
  reverse auctions)

  \textbf{Low/Non-Monotonic Revenue}: Can yield zero revenue despite
  high values (e.g., two identical bidders cancel externalities);
  revenue increases if bidders drop out or reduce bids, violating monotonicity.

  \textbf{Computational Challenges}: Solving $w^*$ often NP-hard
  (e.g., combinatorial auctions ≈ knapsack).
}
\subsection{Open Problems in Auction Design}

VCG guarantees DSIC, efficiency, and non-negative utilities but suffers:

- Low revenue (prioritizes efficiency over seller profit).

- Collusion (shills, bid coordination).

- Computational intractability for large $C$.

- Alternatives trade efficiency for revenue (e.g., Myerson optimal
auctions maximize expected revenue under Bayesian priors).

\examBox{These align with exam themes} {
  - equilibria (Nash vs. dominant)

  - strategies (truthful vs. strategic)

  - repeated/repeated-like manipulations

  - efficiency in strategic settings
}

