\section{Auctions}

- $N$ Players: bidders

- Action: bid $x_i ≥ 0$

\textbf{Outcome}

- $w(x)$ the winner of the auction

- $p(x)$ the price that the winner has to pay

- $t_i$ the true value of the item for agent $i$

The outcome (cost) for agent $i$ is:
\[J_i =
  \begin{cases}
    p(x) − t_i & \text{if} i = w(x)\\
    0& \text{otherwise}
  \end{cases}
\]

\subsection{First-price auctions}

In first-price auctions,
the best-response bid depends
on the bids of the other agents
and your own true value.

- Underbidding $x_i< t_i$

- Truthful bidding $x_i= t_i$

- Overbidding $x_i> t_i$

\begin{proposition}[first-price auctions].\newline
  \begin{itemize}
    \item
      any \textbf{overbidding} strategy is dominated by
      \textbf{truthful bidding}
    \item
      \textbf{truthful bidding} is not a dominant strategy
  \end{itemize}
\end{proposition}

\subsection{The problem of private information}

Until now, we always assumed that the cost functions
of the agents is known to the other agents.
Auctions are an application of game theory
in which this is not true, and we encode
all the \textbf{private information} in the true value $t_i$.

Lack of information leads to inefficiency!

\subsection{Second-price auctions}

\textbf{Winner selection}
$w(x) = \argmax_i x_i$

(the bidder with the highest bid wins)

\textbf{Payment rule}
$p(x) = \max_{i \neq w(x)} x_i$

(the winner pays the second-largest bid)

\subsubsection{Dominant Strategy}

In second-price auctions,
a Nash equilibrium exists and
\textbf{ can be computed by each agent
based on their own private information.}

\textbf{Truthful bidding}
is a weakly dominant strategy
in a second-price auction

Intuition:
your bid determines whether you win,
not how much you pay.

% AI: explain:
- Incentive compatibility

\subsubsection{Properties of second-price auction}

% AI: explain:
Social efficiency

Does the auctioneer achieve the highest return?

No (cost of eliciting truthful bidding...)

\textbf{Incentive compatibility} and \textbf{social efficiency}
often go together
(the true value needs to be disclosed
in order to be used for efficient allocation).

\subsection{Generalized auctions}

\subsubsection{Bids}
Each \textbf{bid} is represented by a pair
$x_j = (b_j, m_j)$

- $b_j$ is the bidded amount

- $m_j$ describes the object of the bid

(Can be extended to allow multiple bids)

\textbf{Fungible goods}
$m_j\in\mathbb{R}_{>0}$
parts of a total quantity $M$

\textbf{Non-fungible goods}
$m_j\in 2^\mathcal{M}$
with finite set of items $\mathcal{M}$

\subsubsection{Choice function}

Choice function $w$ maps bids $x$ into
$N$-dimensional binary vector
\[w_j(x) =
  \begin{cases}
    1 & \text{if bid }j\text{ is accepted}\\
    0& \text{otherwise}
  \end{cases}
\]

- Choice constraints

\subsubsection{Payment function}

Payment function $p$ maps bids $x$ into
$N$-dimensional vector where $p_j(x)$
is the payment requested from the player
that placed the bid $j$

\subsection{VCG auctions}

% AI: function definition
- VCG choice function

% AI: function definition
- VCG payment function

In order to compute each payment $p_j$,
we need to evaluate the choice function twice:
with and without the bid $j$

\subsubsection{Social utility}

The social utility is the aggregate utility
of all players and the auctioneer
$$
U(t,w) = \sum_i t_i w_i
$$

The social utility
- depends on the true value of the goods
according to the player that receives it

- does not depend on the entity of the payments

\textbf{If players bid truthfully} $b_j = t_j$
then VCG choice function achieves
maximal social utility $U^\star$

% AI: explain
Interpretation of the VCG payment if agents bid truthfully:

\subsubsection{Non-negative utility}

% AI: proof
When an agent bids truthfully, his utility is non-negative.

\subsubsection{Dominant bidding strategy}

% AI: proof
Truthful bidding is a weakly dominant strategy in a VCG auction.

\subsection{Open problem of auction design}

We saw how to design an auction which guarantees

- incentive compatibility / truthful bidding

- optimal social efficiency

- non-negative payments

Unfortunately, it comes with drawbacks. For example

- it yields low returns

- it can be manipulated by colluding agents

- it is computationally challenging to solve.

