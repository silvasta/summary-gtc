\section{Repeated games}

\subsection{Tragedy of the commons}

$N$ players (companies, countries) have access to a single resource

Players can

- \textbf{cooperate} and limit their consumption

- \textbf{exploit} the resource, leading to overuse and depletion

2 Players (maximizers)
\[
  \begin{matrix}
    \begin{matrix}
    \end{matrix}&
    \begin{matrix}
      \text{cooperate}& \text{exploit}
    \end{matrix}\\
    \begin{matrix}
      \\ \text{cooperate} \\ \text{exploit}
    \end{matrix}&
    \begin{bmatrix}
      (a,a)&(0,b)\\
      (b,0)&(c,c)
    \end{bmatrix}
  \end{matrix}
\]

N Players
\[
  P_i =
  \begin{cases}
    a\frac{N_C -1}{N-1} \text{(cooperate)}
    \\
    (b-c)\frac{N_C}{N-1}+c \text{(exploit)}
  \end{cases}
\]

Best response:

Exploit

- is a dominant strategy, and therefore the pure strategy

- is the only Nash Equilibrium

\subsubsection{Application domains}

\begin{itemize}
  \item Human population growth leading to overpopulation
  \item Atmospheric pollution (ozone depletion, global warming, ...)
  \item Water use and overirrigation
  \item Logging of old forests
  \item Fossil fuel use and consequential global warming
  \item Ocean overfishing
  \item Antibiotic use and antibiotic resistance
  \item Vaccinations and herd immunity
  \item Wi-Fi channel use and transmission power
  \item Hoarding of items such as toilet paper during a perceived threat
\end{itemize}

\subsection{Tragedy of the commons: multi-stage version}

- \textbf{Supergame}
$G^{(T)}$, same game $G$ repeatet $T$ times.

- \textbf{Actions}
$u^t\in$ \{cooperate,exploit\}$^N$ of $N$ players at stage $t$

- \textbf{History of the game}
$\mathcal{H}^t$ actions of $N$ players before stage $t$

$\mathcal{H}^1 = ()$ empty vector,
$\mathcal{H}^T = (u^1,\dots,u^{T-1})$

\textbf{Assumption:}
All agents have perfect recall of the past.

\subsection{Strategies in a repeated game}

- \textbf{Pure strategy}
$\gamma_i^{(T)}$ of Player $i$ is \textbf{sequence of functions}
\[
  \gamma_i^{(T)}:
  \mathcal{H}^t \to \mathcal{U}
  :=\{\text{cooperate},\text{exploit}\}
\]

- \textbf{Payoff} of repeated game:
\[
  P_i^{(T)}(
    \gamma_i^{(T)},\dots, \gamma_N^{(T)}
  )=
  \sum_{t=1}^T P_i(u^t)
\]
- $P_i$ payoff function of the game played at every stage

- $u^t = \gamma^t(\mathcal{H}^t)$

NE
% TODO: Nash equilibrium of the repeated game

Player $i$ cannot improve by unilaterally changing his strategy,
while other agents maintain the same \textbf{strategy} (not \textbf{actions}!)

% NOTE: look at example L7.6/20

\begin{sstTitleBox} {
    Nash equilibria of repeated games
  }
  \begin{theorem}
    Consider a game G, and its repetition T times G(T ) .

    - If

    - If
  \end{theorem}
\end{sstTitleBox}

\subsection{Infinitely repeated games}

Infinite repetition $G^(\infty)$ of the same game $G$.

\begin{sstTitleBox} {
    Discounted payoff
  }
  \[
    P_i^{(\infty)}( \gamma_i^{(\infty)})=
    (1-\delta)\sum_{t=1}^\infty \delta^{t-1}P_i(u^t)
  \]
  with \textbf{discoun factor} $0<\delta<1$
\end{sstTitleBox}

- \textbf{evil} strategy $\gamma_i^t=$ exploit

Proposition

The strategy evil $\forall i$ is a Nash Equilibrium.

- \textbf{trigger} strategy
\[
  \gamma_i^t =
  \begin{cases}
    \text{cooperate}& \text{if} u_j^\tau = \text{cooperate}
    \forall J \neq i, \forall \tau<t
    \\
    \text{exploit}& \text{else}
  \end{cases}
\]

Proposition

NE if...
% NOTE: example with numbers

\textbf{Positive message:}
Cooperation can be achieved via repetition!

\begin{sstTitleBox} {
    Cooperation via repetition
  }
  Consider a static game $G$ and its infinite repetition $G^({\infty})$

  Consider a strategy $\hat{\gamma}$ for $G$
  that achieves individual payoffs
  \[
    \hat{P_i}= P_i(\hat{\gamma}) >
    \bar{P_i}:=
    \min_{\gamma_{-i}} \max_{\gamma_{i}} P_i( \gamma_{-i}, \gamma_{i})
  \]
  Then there exists $\delta_0\in(0,1)$,
  such that for all $\delta\in(\delta_0,1)$
  the game $G^({\infty})$ has a Nash Equilibrium $\gamma^{(\infty)}$
  with $P_i^{(\infty)} = \hat{P_i}$
\end{sstTitleBox}

Infinitely repeated games tend to have \textbf{many Nash equilibria}

