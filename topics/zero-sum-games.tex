\section{Zero-sum games}

Zero-sum games model competitive scenarios where one player's gain is
the other's loss, applicable in areas like security, economics, and
control. Nash equilibria in these games are computationally tractable
and possess unique values, making them foundational for game theory.

\subsection{Two-Person Zero-sum Games}

Games in which the two players
have opposite payoffs:
\[
  J_1 (\gamma,\sigma) = −J_2 (\gamma,\sigma)
\]
Like a static game with Payoff Matrix $B = −A$

Row player (P1, minimizer) pays \(a_{ij}\)
to column player (P2, maximizer).
P1 minimizes expected outcome \(V\), P2 maximizes it.

\subsubsection{Example: Rock-Paper-Scissors}

\subsubsection{Rock, Paper, Scissors}

Consider only one round of the game.

\[
  A =
  \begin{matrix}
    \begin{matrix}
      \begin{matrix}
        & \text{\scriptsize Rock} & \text{\scriptsize Paper} &
        \text{\scriptsize Scissors}
      \end{matrix}\\
      \begin{bmatrix}
        0 & 1 & -1 \\
        -1 & 0 & 1 \\
        1 & -1 & 0 \\
      \end{bmatrix}\\
    \end{matrix}&
    \begin{matrix}
      \\ \text{\scriptsize Rock}\\ \text{\scriptsize
      Paper}\\ \text{\scriptsize Scissors}
    \end{matrix}
  \end{matrix}
\]

No pure Nash equilibrium exists.

\subsection{Security Levels and Policies}

\begin{tabular}{c|c|c}
  & Security level &Security policy \\
  \hline
  P1 &
  \(\bar{V} = \min_i \max_j a_{ij}\)&
  \(\bar{i} \in \arg\min_i (\max_j a_{ij})\)
  \\
  P2 &
  \(\underline{V} = \max_j \min_i a_{ij}\)&
  \(\bar{j} \in \arg\max_j (\min_i a_{ij})\)
\end{tabular}

\subsubsection{Min-Max Property}

Security levels in \textbf{zero-sum games}
have a fundamental property
that general static games don’t have.

For every finite matrix A, the following properties hold:

(i) Security levels are well defined and unique

(ii) Both players have security policies (not necessarily unique)

(iii) The security levels always satisfy:
\(\underline{V} \leq \bar{V}\)

\textbf{Proof} For any matrix \(A\),
\[
  \max_j \min_i a_{ij} \leq \min_i \max_j a_{ij}
\]
as the max of mins cannot exceed the min of maxes.

\subsection{Nash equilibrium in zero-sum games}

A pure Nash equilibrium (saddle-point) \((i^*, j^*)\) satisfies:
\[
  a_{i^* j} \leq a_{i^* j^*} \leq a_{i j^*} \quad \forall i, j.
\]

Known as \textbf{saddle-point equilibrium} with value
$V^\star := a_{i^\star j^\star}$

Interpretation: No regret, stable under iteration.

\subsection{Saddle-point and security levels}

Not all zero-sum games have a saddle point
(f.e. Rock-Paper-Scissors).
We can exactly characterize the zero-sum games
that have a saddle point.

\begin{theorem}[Saddle-point and security levels]
  A zero-sum game defined by $A$
  has a saddle-point equilibrium \textbf{if and only if}
  % TODO: formula min,max max min? (for V_^-)
  \[
    \underbar{V} = \bar{V}
    \quad(=V^\star\ \text{saddle-point value})
  \]

\end{theorem}

\textbf{Proof}: If equal, security
strategies form a saddle-point; conversely, saddle-point implies
equal security levels. Consequences: Unique value \(V^*\) for all
equilibria; interchangeability of strategies.

\textbf{Important consequences follow (only for zero-sum games!)}

All saddle-point equilibria (Nash equilibria)
of a zero-sum game have the same value $V^\star$,
which we denote as the \textbf{value of the game}.

\subsection{Mixed strategies}

\subsubsection{Mixed strategies}

P1 chooses \(y \in \Delta^m\) (simplex),

P2 chooses \(z \in \Delta^n\)

Expected payoff: \(v(y,z) = y^T A z\).

\subsubsection{Mixed security levels}

\(\bar{V}_m = \min_y \max_z y^T A z = \min_y \max_j (A^T y)_j\)

\(\underline{V}_m = \max_z \min_y y^T A z = \max_z \min_i (A z)_i\).

\subsubsection{Computing via Linear Programming}
For P1's mixed security level:
\[
  \min_{y,t} t \quad s.t.\quad A^T y \leq t \cdot \mathbf{1}, \quad
  \mathbf{1}^T y = 1, \quad y \geq 0.
\]
Optimal \(t^* = \bar{V}_m\). Dual gives P2's strategy.

\subsubsection{Min-Max Theorem (von Neumann)}

For finite zero-sum games:
\(\underline{V}_m = \bar{V}_m\) (unique game value).

Always exists mixed Nash equilibrium.

\subsubsection{Nash Equilibrum}

\subsubsection{Mixed Nash Equilibrium}

\begin{theorem}[Mixed Nash equilibrium for zero-sum games]
  Policy $(y^\star,z^\star)$ is called
  mixed-strategy saddle-point equilibrium
  (or Nash equilibrium) if
  \[
    {y^\star}^T A z^\star
    \le
    y^T A z^\star
    \quad\forall y\quad
    \text{(minimizer)}
  \]
  \[
    {y^\star}^T A z^\star
    \ge
    {y^\star}^T A z
    \quad\forall z\quad
    \text{(maximizer)}
  \]
  ${y^\star}^T A z^\star$ is called
  \textbf{saddle point value}
\end{theorem}

\examBox{Exam-Relevant Extensions and Exercises}{
  From exercises: Skew-symmetric games (e.g., "Pick a Number") have
  value 0; mixed NE computed via LP, often mixing few strategies (e.g.,
  uniform over subset).

  Exam connections (e.g., 2024 exam): Extend to extensive-form zero-sum
  games with imperfect information, feedback vs. non-feedback,
  behavioral vs. mixed strategies (Kuhn's theorem: equivalence in
  perfect recall), backward induction for subgame equilibria, and
  computing mixed NE in trees.

  Key properties for exams: Unique value, LP solvability, no pure NE
  implies mixed (e.g., Rock-Paper-Scissors: uniform \(1/3\)); in
  rectangular games, minimizer mixes at most as many as maximizer's actions.
}

