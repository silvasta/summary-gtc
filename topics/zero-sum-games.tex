\section{Zero-sum games}

- Zero-sum games model a large number of practical applications

- Nash equilibria in zero-sum games have many useful properties

- Nash equilibria in zero-sum games are much easier to compute

\subsection{Two-Person Zero-sum Games}

Games in which the two players
have opposite payoffs:
\[
  J_1 (\gamma,\sigma) = −J_2 (\gamma,\sigma)
\]
Static game with $B = −A$
(we only indicate one matrix, $A$)

\subsubsection{Payoff Matrix}

- Row player loses aij

- Column player gains aij

- Row player minimizes outcome V

- Column player maximizes outcome V

\subsubsection{Rock, Paper, Scissors}

Consider only one round of the game.

\[
  A =
  \begin{matrix}
    \begin{matrix}
      \begin{matrix}
        & \text{\scriptsize Rock} & \text{\scriptsize Paper} &
        \text{\scriptsize Scissors}
      \end{matrix}\\
      \begin{bmatrix}
        0 & 1 & -1 \\
        -1 & 0 & 1 \\
        1 & -1 & 0 \\
      \end{bmatrix}\\
    \end{matrix}&
    \begin{matrix}
      \\ \text{\scriptsize Rock}\\ \text{\scriptsize
      Paper}\\ \text{\scriptsize Scissors}
    \end{matrix}
  \end{matrix}
\]

\subsection{Security levels and policies}

- Security level Player 1,2

% AI: extend with definiton

- Security policy Player 1,2

% AI: extend with definiton

\subsubsection{Min-Max Property}

Security levels in \textbf{zero-sum games}
have a fundamental property
that general static games don’t have.

For every finite matrix A, the following properties hold:

(i) Security levels are well defined and unique

(ii) Both players have security policies (not necessarily unique)

(iii) The security levels always satisfy:
% AI: Formula with V-under-over bar depending on Player 1,2

% AI: proof of V_under<V_over forall A

\subsection{Nash equilibrium in zero-sum games}

% AI: definition

\textbf{Intepretation}:
As in static games:
no regret, stable strategy when iterated, etc.

Also known as \textbf{saddle-point equilibrium}
\[
  a_{i^\star j} \le
  a_{i^\star j^\star} \le
  a_{i j^\star}
  \quad
  \forall i \in {1, . . . , n},
  \forall j \in {1, . . . , m}
\]
Saddle-point value $V^\star := a_{i^\star j^\star}$

\subsection{Saddle-point and security levels}

Not all zero-sum games have a saddle point
(f.e. Rock-Paper-Scissors).
We can exactly characterize the zero-sum games
that have a saddle point.

\begin{theorem}[Saddle-point and security levels]
  A zero-sum game defined by $A$
  has a saddle-point equilibrium \textbf{if and only if}
  % TODO: formula min,max max min? (for V_^-)
  \[
    \underbar{V} = \bar{V}
    \quad(=V^\star\ \text{saddle-point value})
  \]
  % AI: proof the theorem
\end{theorem}

\textbf{Important consequences follow (only for zero-sum games!)}

All saddle-point equilibria (Nash equilibria)
of a zero-sum game have the same value $V^\star$,
which we denote as the value of the game.

\subsection{Mixed strategies}

% HACK: Recap L2.15.. in chapter 1??

\subsubsection{Computing the mixed security level via LP}

%AI: Formulas from V_m = min max sum ... to LP

pseudo, matlab code:

% D e f i n e t h e m a t r i x d e f i n i n g t h e zero −sum game
m = 5; n = 10;
A = rand (m, n ) ;
% Define o p t i m i z a t i o n v a r i a b l e s
y = sdpvar (m, 1 ) ;
t = sdpvar ( 1 , 1 ) ;
% Define o b j e c t i v e f u n c t i o n
obj = t ;
% Define c o n s t r a i n t s
c o n s t r a i n t s = [ A ' ∗ y <= ones ( n , 1 ) ∗ t , sum ( y )
== 1 , y >= 0 ] ;
% Solve o p t i m i z a t i o n problem
optimize ( constraints , obj ) ;
% Get s o l u t i o n
S e c u r i t y L e v e l = double ( t ) ;
S e c u r i t y P o l i c y = double ( y ) ;

\subsubsection{Min-Max Property}

% AI: show property for mixed strategies

\subsubsection{Nash Equilibrum}

\begin{theorem}[Mixed Nash equilibrium for zero-sum games]
  Policy $(y^\star,z^\star)$ is called
  mixed-strategy saddle-point equilibrium
  (or Nash equilibrium) if
  \[
    {y^\star}^T A z^\star
    \le
    y^T A z^\star
    \quad\forall y\quad
    \text{(minimizer)}
  \]
  \[
    {y^\star}^T A z^\star
    \ge
    {y^\star}^T A z
    \quad\forall z\quad
    \text{(maximizer)}
  \]
  ${y^\star}^T A z^\star$ is called
  \textbf{saddle point value}
\end{theorem}

